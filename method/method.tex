\section{Method}
\label{sec:method}

% Include exp_setup.jpg figure
\begin{figure}
    \centering
    \includegraphics[width=0.8\textwidth]{method/labeled_diagram.jpg}
    \caption{Diagram of Experimental Setup}
    \label{fig:exp_setup}
\end{figure}

The experiment is divided into three sections:

\subsection{Part 1: Magnetic field strength and current}
In the first part of the experiment, we measure the magnetic field strength as a function of the current supplied to the electromagnet. We do this by measuring the magnetic field strength at the center of the electromagnet using a Hall probe and the current as the current from the power source. We measure the magnetic field strength at 10 different current values, and then fit the data to a linear model to determine the relationship between the current and the magnetic field strength.

\subsection{Part 2: Observing the Zeeman split perpendicularly to the magnetic field with different field values}

In the first part of the expeirment we measure how the Zeeman split changes with different magnetic field values.
We do this by measuring the width of the spectral lines of the mercury lamp with and without a magnetic field.
We then calculate the Zeeman shift for each trial.

First, without the filed turned on, we measure the position given by the dial gauge for a number of lines.
Then, we turn on the magnetic field and measure the upper and lower bounds of each line. Because
the lines do not appear to perfectly separate, and only a widening of the lines can be observed,
we measure the top most and bottom most points of the line, and calculate the width of the line as the difference between the two.

The expected result is that the shift will be proportional to the field.