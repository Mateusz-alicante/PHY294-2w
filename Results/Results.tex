
\section{Results}

After adjusting the magnet poles, several measurments of the magnetic filed intensity were taken using the Gauss meter. The results are shown in the table below.

\begin{table}
    \begin{tabular}{|l|l|l|l|l|l|}
        \hline
        {\ul \textbf{Trial:}}        & {\ul Trial 1} & {\ul Trial 2}  & {\ul Trial 3}  & {\ul Trial 4}  & {\ul Trial 4} \\ \hline
        {\ul \textbf{Field Value:} } & $595 \pm 0.5$ & $604  \pm 0.5$ & $600  \pm 0.5$ & $588  \pm 0.5$ & $601 \pm 0.5$ \\ \hline
    \end{tabular}
    \caption{Magnetic Field Intensity in mT}

\end{table}

These measurments result in an average value of: $B = 598 \pm 0.22$ mT.

The field current and pole positions were not changed throughout the experiment, so we assume the
magnetic field intensity to be constant.

\subsection{Part 1: Observing the Zeeman split perpendicularly to the magnetic field}

% Source: Raw_Data_Part_1.tgn
\begin{table}[]
    \begin{tabular}{|l|
            >{\columncolor[HTML]{34FF34}}l |
            >{\columncolor[HTML]{34CDF9}}l |
            >{\columncolor[HTML]{34CDF9}}l |}
        \hline
        {\ul }      & Without Field & With Field          &                     \\ \hline
        {\ul Trial} & Line Position & Upper edge position & Lower edge position \\ \hline
        {\ul 1}     & 3.61          & 3.67                & 3.60                \\ \hline
        {\ul 2}     & 3.44          & 3.55                & 3.46                \\ \hline
        {\ul 3}     & 3.29          & 3.40                & 3.34                \\ \hline
        {\ul 4}     & 3.15          & 3.27                & 3.18                \\ \hline
        {\ul 5}     & 3.03          & 3.14                & 3.06                \\ \hline
        {\ul 6}     & 2.90          & 3.00                & 2.94                \\ \hline
    \end{tabular}
    \caption{Raw Data for Part 1}
\end{table}